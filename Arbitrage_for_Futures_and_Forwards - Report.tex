\documentclass[12pt]{article}

\usepackage[a4paper, margin=1in]{geometry}
\usepackage{amsmath}
\usepackage{amssymb}

\title{Multi-Asset Arbitrage Strategy Framework in Excel\\
Stocks, Commodities and Currencies}
\author{Adam El Gbouri}
\date{January 2026}

\begin{document}

\maketitle

\section{Project Overview}

This project consists of the development of a multi-asset arbitrage detection framework implemented in Excel. The model covers three asset classes:

\begin{itemize}
    \item Equities (Stock index or single stock futures)
    \item Commodities
    \item Foreign Exchange (FX forwards)
\end{itemize}

For each asset class, the model:

\begin{itemize}
    \item Computes the theoretical forward or futures price
    \item Measures mispricing relative to the observed market price
    \item Determines whether an arbitrage opportunity exists
    \item Automatically generates the appropriate trading strategy
    \item Computes the associated profit and loss (PnL)
\end{itemize}

All arbitrage decisions are determined dynamically from the sign of the mispricing variable.

\section{General Arbitrage Logic}

For each asset class, mispricing is defined as:

\[
\text{Mispricing} = F_{\text{market}} - F_{\text{theoretical}}
\]

The trading strategy is entirely driven by the sign of this value:

\begin{itemize}
    \item If mispricing $> 0$: the contract is overpriced
    \item If mispricing $< 0$: the contract is underpriced
    \item If mispricing $= 0$: no arbitrage opportunity
\end{itemize}

The PnL corresponds to the theoretical risk-free gain obtained by implementing the strategy until maturity.

\section{Equity Arbitrage (Cash-and-Carry Framework)}

\subsection*{Theoretical Pricing}

For equities paying a continuous dividend yield:

\[
F_0 = S_0 e^{(r - q)T}
\]

where:

\begin{itemize}
    \item $S_0$ is the spot price
    \item $r$ is the risk-free rate
    \item $q$ is the dividend yield
    \item $T$ is time to maturity
\end{itemize}

\subsection*{Arbitrage Strategies}

\textbf{If mispricing $> 0$ (Future Overpriced):}

\begin{itemize}
    \item Buy the Spot (or ETF)
    \item Sell the Future
    \item Hold until maturity
    \item Deliver spot / settle the future
\end{itemize}

\textbf{If mispricing $< 0$ (Future Underpriced):}

\begin{itemize}
    \item Short the Spot (or ETF)
    \item Buy the Future
    \item Hold until maturity
\end{itemize}

\textbf{If mispricing $= 0$:}

\begin{itemize}
    \item No arbitrage possible
\end{itemize}

\section{Commodity Arbitrage (Cost-of-Carry Model)}

\subsection*{Theoretical Pricing}

For commodities:

\[
F_0 = S_0 e^{(r + u - y)T}
\]

where:

\begin{itemize}
    \item $u$ represents storage, insurance, financing and transport costs
    \item $y$ represents convenience yield
\end{itemize}

In the model:

\begin{itemize}
    \item Cost of carry is computed dynamically
    \item Convenience yield is derived via an Excel formula
\end{itemize}

\subsection*{Arbitrage Strategies}

\textbf{If mispricing $> 0$ (Future Overpriced):}

\begin{itemize}
    \item Buy the physical commodity
    \item Pay storage + insurance + financing + transport
    \item Sell the Future
    \item Deliver at maturity
\end{itemize}

\textbf{If mispricing $< 0$ (Future Underpriced):}

\begin{itemize}
    \item Short the physical commodity
    \item Buy the Future
\end{itemize}

\textbf{If mispricing $= 0$:}

\begin{itemize}
    \item No arbitrage possible
\end{itemize}

\section{Foreign Exchange Arbitrage (Covered Interest Arbitrage)}

\subsection*{Theoretical Pricing}

Under Covered Interest Parity (CIP):

\[
F_0 = S_0 e^{(r_d - r_f)T}
\]

where:

\begin{itemize}
    \item $r_d$ is the domestic interest rate
    \item $r_f$ is the foreign interest rate
\end{itemize}

\subsection*{Arbitrage Strategies}

\textbf{If mispricing $> 0$ (Forward Overpriced):}

\begin{itemize}
    \item Borrow foreign currency
    \item Convert to domestic currency (spot)
    \item Invest domestically
    \item Sell the forward to lock future conversion
\end{itemize}

\textbf{If mispricing $< 0$ (Forward Underpriced):}

\begin{itemize}
    \item Borrow domestic currency
    \item Convert to foreign currency
    \item Invest abroad
    \item Buy the forward
\end{itemize}

\textbf{If mispricing $= 0$:}

\begin{itemize}
    \item No arbitrage possible
\end{itemize}

\section{Model Structure}

Each asset class has its own dedicated input structure:

\begin{itemize}
    \item Maturity in years is calculated via Excel formulas
    \item For commodities, cost of carry and convenience yield are dynamically computed
    \item Strategy generation is handled via conditional logic based on the mispricing cell
\end{itemize}

The arbitrage strategy is therefore fully automated and consistent with no-arbitrage pricing theory.

\section{Potential Extensions}

The framework can be extended by:

\begin{itemize}
    \item Integrating live market data feeds (Bloomberg, Refinitiv, APIs)
    \item Including transaction costs and funding spreads
    \item Adding capital allocation constraints
    \item Implementing margin and liquidity considerations
    \item Building a real-time arbitrage monitoring dashboard
    \item Automating asset selection through VBA UserForms
\end{itemize}

\section{Conclusion}

This project implements a unified arbitrage detection engine across equities, commodities, and FX markets. It combines theoretical pricing models with automated strategy generation and direct PnL estimation.

The framework demonstrates practical application of:

\begin{itemize}
    \item Cash-and-Carry and Reverse Cash-and-Carry
    \item Cost-of-Carry pricing
    \item Covered Interest Arbitrage
\end{itemize}

It provides a structured approach to identifying and quantifying cross-asset arbitrage opportunities.

\end{document}